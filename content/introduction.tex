As the landscape of software development continues to evolve, the prevalence of distributed systems has become increasingly widespread.~\cite{Xingang2018,Feldman1978}. This paradigm shift brings forth a new set of challenges, particularly in the realm of testing, as traditional methodologies often struggle to encapsulate the complexities inherent to these distributed architectures~\cite{Liu,Lima2017}. The testing of distributed applications as holistic entities, rather than isolated components, has become a necessary practice to ensure seamless integration and robust functionality~\cite{Liu,Lima2017}.
Unlike unit testing, where individual components are tested in isolation, \acf{BDD} testing adopts a different perspective. It tests from the point of view of the end user, focusing on the overall behavior and functionality of the application~\cite{wynne2012cucumber}. This approach ensures that the system functions cohesively and meets the expectations of the end user~\cite{wynne2012cucumber}.

Traditional testing methodologies, designed for monolithic systems, often fall short in capturing the intricate dependencies and interactions among distributed components~\cite{Liu,Lima2017}. This challenge becomes particularly apparent when considering ~\acf{RCE}, developed by the~\acf{DLR}. \ac{RCE}, as a distributed system, encapsulates the complexities inherent in network applications, demanding a departure from conventional testing norms.


The inherent challenges of testing \ac{RCE} stem from its server-based remote component architecture. Within this framework, \ac{RCE} client instances operate as thin clients, accessing server-provided components and functionalities within a complex calculation and simulation workflow. A \ac{RCE} workflow comprises components with predefined inputs and outputs connected to each other, thereby defining the data flow. This design emphasizes the need for a comprehensive testing approach, as unexpected behavior of remote server instances, coupled with network-shared components, could render the client application useless or lead to inaccurate calculations.

Unlike the conventional approach of testing individual components locally, it is noteworthy that \ac{RCE} currently employs a holistic testing strategy aligned with real-world deployment scenarios. The adoption of a \ac{BDD} testing strategy for \ac{RCE} by the \ac{DLR} represents a significant change towards comprehensive testing methodologies. This approach, using tools such as Gherkin and Cucumber, not only addresses the inherent challenges of testing distributed applications but also underscores the significance of a collaborative and holistic testing paradigm. The key focus lies on the expected behavior and functioning of the application from the perspective of end-users.

By exploring the intricacies of \ac{RCE}'s BDD testing setup, this paper seeks to uncover the multifaceted challenges associated with testing distributed systems. In addition to addressing these challenges, the emphasis is on extracting valuable insights from \ac{RCE}'s testing methodology. Taking \ac{RCE} as a case study, the objective of the article is to shed light on how an all-encompassing testing strategy can competently tackle the distinctive complexities of distributed applications. 
%The findings from this exploration offer valuable lessons that can be applied to the broader realm of software testing within the distributed paradigm.