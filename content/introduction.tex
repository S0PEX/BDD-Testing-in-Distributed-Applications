As the landscape of software development continues to evolve, the prevalence of distributed systems has become increasingly commonplace~\cite{Xingang2018,Feldman1978}. This paradigm shift brings forth a new set of challenges, particularly in the realm of testing, as traditional methodologies often struggle to encapsulate the complexities inherent to these distributed architectures~\cite{Liu,Lima2017}. Testing distributed applications as holistic entities, rather than isolated components, has emerged as a necessary practice to ensure seamless integration and robust functionality~\cite{Liu,Lima2017}.

Traditional testing methodologies, designed for monolithic systems, often fall short in capturing the intricate dependencies and interactions among distributed components~\cite{Liu,Lima2017}. This challenge becomes particularly apparent when considering ~\ac{RCE}, developed by the~\ac{DLR}. \ac{RCE}, as a distributed system, encapsulates the complexities inherent in network applications, demanding a departure from conventional testing norms.

The inherent challenges of testing \ac{RCE} stem from its server-based remote component architecture. Within this architecture, \ac{RCE} client instances operate as thin clients, accessing server-provided components and functionalities within a complex calculation and simulation workflow. A workflow in \ac{RCE} consists of components with predefined inputs and outputs connected to each other, which define the data flow. This design underscores the need for a comprehensive testing approach, as unexpected behavior of remote server instances, combined with network-shared components, could render the client application useless or result in incorrect calculations. Thus, unlike the conventional approach of testing individual components locally, \ac{RCE} demands a holistic testing strategy aligned with the real-world deployment scenarios it is intended to handle. Adopting a \acf{BDD} testing strategy for \ac{RCE} by \ac{DLR} marks a substantial shift toward comprehensive testing methodologies. This approach, utilizing tools like Gherkin and Cucumber, not only addresses the challenges inherent in testing distributed applications but also points out the significance of a collaborative and holistic testing paradigm, where the key focus lies on the expected behavior and functioning of the application from the view of their end-users.

By exploring the intricacies of \ac{RCE}'s BDD testing setup, this paper seeks to uncover the multifaceted challenges associated with testing distributed systems. In addition to addressing these challenges, the emphasis is on extracting valuable insights from \ac{RCE}'s testing methodology. Taking \ac{RCE} as a case study, the objective of the article is to shed light on how an all-encompassing testing strategy can competently tackle the distinctive complexities of distributed applications. 
%The findings from this exploration offer valuable lessons that can be applied to the broader realm of software testing within the distributed paradigm.