As the landscape of software development continues to evolve, the employment of distributed systems has become increasingly widespread.~\cite{Xingang2018,Feldman1978}. This paradigm shift brings forth a new set of challenges, particularly in the realm of testing, as traditional methodologies often struggle to encapsulate the complexities inherent to these distributed architectures~\cite{Liu,Lima2017}. The testing of distributed applications as holistic entities, rather than isolated components, has become a necessary practice to ensure seamless integration and robust functionality~\cite{Liu,Lima2017}.

Traditional testing methodologies, designed for monolithic systems, often fall short in capturing the intricate dependencies and interactions among distributed components~\cite{Liu,Lima2017}. This challenge becomes particularly apparent when considering ~\acf{RCE}, developed by the~\acf{DLR}. \ac{RCE}, as a distributed system, encapsulates the complexities inherent in network applications, demanding a departure from conventional testing norms.

Within \ac{RCE}'s server-based remote component architecture, client instances operate as thin clients, accessing server-provided components and functionalities within a complex calculation and simulation workflow. A \ac{RCE} workflow comprises components with predefined inputs and outputs connected to each other, thereby defining the data flow. This design emphasizes the need for a comprehensive testing approach, as unexpected behavior of remote server instances, coupled with network-shared components, could render the client application useless or lead to inaccurate calculations.

Unlike the conventional approach of testing individual components locally, \ac{RCE} employs a testing strategy aligned with real-world deployment scenarios, implementing a \ac{BDD} testing strategy. This \ac{BDD} methodology emphasizes the application's complete behavior and functionality from the user's viewpoint, guaranteeing that the system operates seamlessly and fulfills end user expectations~\cite{wynne2012cucumber}. Using tools such as the \ac{DSL} Gherkin to formulate requirements in natural language (~\Cref{subsec:gherkin}) and Cucumber to link these requirements to actual test cases (~\Cref{subsec:cucumber}), this approach does not only address the challenges of testing distributed applications, but also underscores the significance of a collaborative and holistic testing paradigm.

By extracting valuable insights about \ac{RCE}'s testing methodology and exploring ways to enhance them, this paper aims to provide actionable suggestions for strengthening the \ac{BDD} testing setup used in \ac{RCE}.We provide necessary background on fundamental concepts in \Cref{sec:background} and our research methodology in \Cref{sec:method}. We then proceed to examine the \ac{RCE} project and present our findings in \Cref{sec:examination}, subsequently offering possible solutions in \Cref{sec:results}. These solutions are evaluated in \Cref{sec:discussion}and a brief summary of our work is given in \Cref{sec:conclusion}.