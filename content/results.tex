In this section, we present a concise summary of our findings derived from the thorough examination of \ac{RCE} that is presented in ~\Cref{sec:examination}. Initially, we revisit our observations concerning the current setup of RCE, highlighting the identified issues and challenges in~\Cref{sub:identified-problems}. Subsequently, we delve into the aspect of missing network fault tests, shedding light on the implications and gaps observed in this critical area of RCE's testing framework in~\Cref{sub:lack-network-tests}. 

\subsection{Identified problems in RCE's Testing Setup}
\label{sub:identified-problems}
In this section, we will address identified problems in RCE's testing setup, structuring our discussion around three key points: the presence of static waits, the intricate use of complex regex patterns, and the lack of network tests simulating faults.

\subsubsection{Static WAITs}
In numerous scenarios within \ac{RCE}'s testing setup, processes such as the execution of workflows or ensuring a specific system state rely on static waits. Our analysis, particularly in~\Cref{subsec:fault-tolerance}, highlighted the limitation of this approach, especially when the system environment requires more time than the predefined static wait duration. This mismatch can lead to unnecessary test failures. To address this, we recommend the adoption of dynamic waits or conditional waits where feasible. While dynamic waits with timeouts might be suitable in cases where checking for workflow completion is challenging, implementing these adaptive strategies enhances the resilience of tests to variable system conditions. Alternatively, the method of timeouts, already used in the Network Tests (e.g., \verb|@Network02| with the ``THEN``-clause \verb|...should be ready within 20 seconds|), can be applied. This approach employs periodic checks within a specified timeframe, serving as a timeout implementation that adapts to varying system conditions.


\subsubsection{Suboptimal Application of Regex in Cucumber Step Definitions}
\label{subsec:resultCucumber}

In~\Cref{subsec:CodeReview}, we highlight the challenges posed by the use of \ac{Regex} in Cucumber annotations, particularly in terms of their impact on readability and simplicity. The intricate syntax of \ac{Regex}, while offering robust pattern matching functionality, tends to result in complex and difficult to interpret annotations, especially for individuals who lack proficiency in \ac{Regex}. The introduction of Cucumber Expressions offers a solution to the identified issues, providing a simpler and more intuitive syntax that enhances readability and comprehensibility, thereby alleviating cognitive stress while improving overall code maintainability. For an illustrative contrast between the same step definition annotation utilizing \ac{Regex} and equivalent Cucumber Expressions, see~\Cref{lst:simpleRegex}.

\begin{listing}[!ht]
\caption{Code snippet illustrating the same step definition implemented with \ac{Regex} in the upper part and Cucumber Expressions in the lower part.}
\label{lst:simpleRegex}
\inputminted{java}{files/code/simpleRegex.java}
\end{listing}

In this example, the ease of using Cucumber Expressions becomes evident. In contrast to the necessity for users to have a comprehensive understanding of capturing groups, character classes, negation, and quantifiers in order to decipher this particular \ac{Regex}, Cucumber Expressions and their intuitive placeholders (e.g., \verb|{string}, {int}, {float}|, etc.)", offer a starkly different user experience when capturing primitive data types. While such placeholders offer improved readability and ease of use, it is important to acknowledge that they do not possess the same level of expressive power as \ac{Regex}. This distinction introduces a trade-off between simplicity and flexibility, as further discussed in~\Cref{sec:discussion}. 


\subsubsection{Lack of Genuine Network Tests Incorporating Network Faults}
\label{sub:lack-network-tests}

In the current test setup, the network aspect of the application is examined on the localhost by initiating multiple local instances. However, this approach might overlook critical real-world circumstances present in an actual distributed environment. As the current setup lacks the capability to test the application's behavior under conditions like packet loss or connection timeouts, it becomes imperative to implement additional tests focusing on these specific failures. Since there is no built-in method in Java to simulate such failures, external tools like WinDivert (cf. ~\Cref{subsec:windivert}) or Chaos Mesh (cf. ~\Cref{subsec:chaosmesh}) can be utilized.

WinDivert is bound to a Windows system and can intercept network traffic at the kernel level, enabling the introduction of network faults. Although it has Java bindings and can be seamlessly integrated into the existing suite, creating specific network faults would require manual intervention, as the library primarily offers basic methods for monitoring, modifying, or dropping packages. On the other hand, tools like Chaos Mesh, built upon \ac{K8S} (cf. ~\Cref{subsec:kubernetes}), can independently generate various network faults. However, they require the execution of tests within a \ac{K8S} cluster. This, in turn, requires the setup and configuration of either a local \ac{K8S} cluster (e.g., minikube) or the use of a remotely hosted cluster, thus adding to the initial setup complexity of the \ac{RCE} development process. Hampel (2022) provides an extended discussion on the \ac{K8S} approach, wherein chaos scenarios are employed to glean insights into the behavior of \ac{RCE} under unforeseen network conditions~\cite{Hampel2022}.

% Hier windivert und das andere Zeiug als subsec