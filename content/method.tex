In this chapter, we introduce our task and discuss our chosen approach. First, in~\Cref{subsec:Task}, we describe the assignment. Subsequently, in~\Cref{subsec:Methodologies} we discuss the methods we selected to fulfill this assignment.

\subsection{Task Background}
\label{subsec:Task}
In the initial stages, our assignment was explicitly outlined: we were tasked with examining the configuration of \ac{BDD} tests in \ac{RCE}. This investigation aimed to understand the current testing framework used and, later, be able to provide valuable information through a systematic evaluation of this setup. Based on this initial directive, our objective was to successfully compile the \ac{RCE} source code, execute existing tests, and verify their functionality. Subsequently, we should shift our focus to a detailed analysis of the testing code base and feature files, seeking out any errors or areas for improvement. Ultimately, time permitting, the last task on our agenda was aimed at the implementation of improvements within the current code-base.

\subsection{Methodology}
\label{subsec:Methodologies}
Based on the fact that our task could be divided into three subtasks, each pursuing a different objective, we have decided to employ various methods. Therefore, we decided to initially examine and extract information from the existing code to understand and trace the test structure in \ac{RCE} via an initial examination and documentation study.

Given the understanding of the respective technologies, we have decided to conduct an in-depth examination of the existing tests through a code review, coupled with the execution of the current test cases as a black box. This approach has the advantage that, first of all, by simply running the tests, we may discover test cases that yield negative results and point us to broken tests. Recognizing that the accuracy of test outputs is only as reliable as the tests themselves, we have opted to simultaneously perform a code review. This allows us to verify whether the tests are indeed evaluating what they purport to test. 

To address the last point on our agenda, which comprises implementing improvements and introducing new test cases, we have devised a strategy informed by the fact that we are not experts in the domain of \ac{RCE}. Given our limited familiarity with the software, we recognize the importance of adopting an exploratory approach to programming. This methodology proves advantageous in situations where comprehensive domain knowledge is lacking. By engaging in exploratory programming, we aim to iteratively experiment with the software, uncovering potential scenarios and functionalities that may not be apparent through conventional testing methods. This approach allows us to supplement the existing testing framework with additional test cases, thus enhancing its robustness and expanding its coverage.