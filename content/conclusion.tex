\label{sec:conclusion}
In the course of this paper, we have explored the intricacies of the existing RCE testing setup. Our investigation revealed that \acf{BDD} tests are conducted using the Cucumber library and are orchestrated through a component known as \acf{TSR}, which relies on the \acf{IM} to simulate a distributed system on a local machine, as depicted in ~\Cref{fig:rce-setup}. This is achieved by launching multiple RCE instances locally and facilitate communication via the loopback device. 

An in-depth examination of the codebase revealed notable issues, particularly the reliance on static waits and the utilization of intricate regex patterns, as well as the absence of genuine network tests that incorporate common network faults. We have elucidated the potential ramifications of these issues and proposed viable solutions, such as the adoption of dynamic waits and conditional waits for improved adaptability, and the introduction of Cucumber expressions to enhance readability and comprehensibility. Moreover, we proposed the integration of tools like Chaos Mesh, a solution capable of simulating various network faults and resource throttling within a \acf{K8S} environment, to target the missing network failure test. This recommendation positions Chaos Mesh as a versatile tool that not only addresses current requirements, but also offers scalability for potential future testing needs. Alternatively, for scenarios where Windows is the predominant operating system, WinDivert emerges as a feasible solution. Its simplicity and Java compatibility make it an attractive option, though it is limited to Windows environments. 

In conclusion, our comprehensive analysis provides valuable insights into the strengths and areas for enhancement within the RCE testing setup. We believe that addressing these identified issues, could improve the reliability and effectiveness of the testing infrastructure, while also aligning it more closely with real-world testing scenarios. Despite the challenges addressed, the modular and decoupled design of the \ac{IM} and \ac{TSR} components emerged as commendable strengths, improving maintainability and flexibility in the testing architecture.
