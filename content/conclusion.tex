In the course of this paper, we have explored the intricacies of the existing RCE testing setup. Our investigation revealed that \acf{BDD} tests are conducted using the Cucumber library and are orchestrated through a component known as \acf{TSR}, which relies on the \acf{IM} to simulate a distributed system on a local machine. This is achieved by launching multiple RCE instances locally and facilitating communication via the loopback device. 

An in-depth examination of the codebase unearthed notable issues, particularly the reliance on static waits and the utilization of intricate regex patterns, as well as the absence of tests associated with common. We have elucidated the potential ramifications of these issues and proposed viable solutions, such as the adoption of dynamic waits and conditional waits for improved adaptability and the introduction of Cucumber expressions to enhance readability. Furthermore, our exploration uncovered the absence of genuine network tests that incorporate network faults.

A key recommendation is the refactoring of static waits, advocating for the adoption of timeout-based approaches, as exemplified in the network tests, or the utilization of dynamic waits, such as exponential fallbacks. This shift ensures a more adaptive and resilient testing infrastructure.
Moreover, we propose the integration of tools like Chaos Mesh, a comprehensive solution capable of simulating various network faults and resource throttling within a Kubernetes environment, to target the missing network failure test. This recommendation positions Chaos Mesh as a versatile tool that not only addresses current requirements but also offers scalability for potential future testing needs. 

Alternatively, for scenarios where Windows is the predominant operating system, WinDivert emerges as a feasible solution. Its simplicity and Java compatibility make it an attractive option, though limited to Windows environments. Despite these challenges, the modular and decoupled design of the Instance Management and Test Suite Runner components emerged as commendable strengths, enhancing maintainability and flexibility in the testing architecture.

In conclusion, our comprehensive analysis provides valuable insights into the strengths and areas for enhancement within the RCE testing setup. By addressing these identified issues, we aim to fortify the reliability and effectiveness of the testing infrastructure, aligning it more closely with real-world testing scenarios.